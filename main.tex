\documentclass[conference]{IEEEtran}
\IEEEoverridecommandlockouts
% The preceding line is only needed to identify funding in the first footnote. If that is unneeded, please comment it out.
\usepackage{cite}
\usepackage{amsmath,amssymb,amsfonts}
\usepackage{algorithmic}
\usepackage{graphicx}
\usepackage{textcomp}
\usepackage{xcolor}
\usepackage[colorlinks=true, linkcolor=blue, citecolor=blue, urlcolor=blue]{hyperref}
\usepackage{cleveref}

\def\BibTeX{{\rm B\kern-.05em{\sc i\kern-.025em b}\kern-.08em
    T\kern-.1667em\lower.7ex\hbox{E}\kern-.125emX}}
\begin{document}
\bibliographystyle{IEEEtran}

\title{A survey of rural and city road identification via satellite\\
}

\author{\IEEEauthorblockN{1\textsuperscript{st} Filipe Cavalheiro}
\IEEEauthorblockA{\textit{DEEC} \\
\textit{NOVA FCT}\\
Lisbon, Portugal \\
https://orcid.org/0009-0000-1238-3975}
\and
\IEEEauthorblockN{2\textsuperscript{nd} Diogo Novais}
\IEEEauthorblockA{\textit{DEEC} \\
\textit{NOVA FCT}\\
Lisbon, Portugal \\
dm.novais@camppus.fct.unl.pt}
}

\maketitle

\begin{abstract}
This paper has a main purpose of describing the main methods of road detection for rural roads and city roads, comparing all methods and stating pros and cons of each one.
\end{abstract}

\begin{IEEEkeywords}
CNN
\end{IEEEkeywords}

\section{Introduction}
THIS SECTION WAS STOLEN AND NEEDS TO BE MODIFIED\\
Road building has long been under-mapped globally, arguably more than any other human activity threatening environmental integrity. Millions of kilometers of unmapped roads have
challenged environmental governance and conservation in remote frontiers. Prior attempts to map
roads at large scales have proven inefficient, incomplete, and unamenable to continuous road monitoring. Recent developments in automated road detection using artificial intelligence have been
promising but have neglected the relatively irregular, sparse, rustic roadways characteristic of remote
semi-natural areas. In response, we tested the accuracy of automated approaches to large-scale road
mapping across remote rural and semi-forested areas of equatorial Asia-Pacific. Three machine learning models based on convolutional neural networks (UNet and two ResNet variants) were trained
on road data derived from visual interpretations of freely available high-resolution satellite imagery.
The models mapped roads with appreciable accuracies, with F1 scores of 72-81\% and intersection
over union scores of 43-58\%. These results, as well as the purposeful simplicity and availability of
our input data, support the possibility of concerted program of exhaustive, automated road mapping
and monitoring across large, remote, tropical areas threatened by human encroachment.


\section{Know methods of error measurment}
Mean Intersection over Union Metric of Model Accuracy:
\begin{equation}
    \text{mIoU} = \frac{1}{N}\sum_{i=0}^{N}\frac{\text{Predicted Road}\cap\text{Known Road}}{\text{Predicted Road}\cup \text{Known Road}}
    \label{eq:mIoU}
\end{equation}

F1 Score of Model Accuracy:
\begin{equation}
    \text{F1 Score} = \frac{1}{N}\sum_{i=0}^{N} \frac{2 X \text{Area of overlap}}{\text{Total Area}}
    \label{eq:F1_score}
\end{equation}

\section{AI MODELS}
\cite{rs16050839} UNet Model; ResNet-34 Model; Resnet-34 Model with Added Residual Connections (ResNet-34+)  The models ResNet-34 and ResNet-34+ performed better than UNet. The methods use to calculate error of the models where: \cref{eq:mIoU,eq:F1_score}.

\bibliography{references}
\end{document}
